% latex table generated in R 4.4.1 by xtable 1.8-4 package
% Tue Sep 24 06:28:59 2024
\begingroup\fontsize{8pt}{10pt}\selectfont
\begin{longtable}{|p{0.08\textwidth}|p{0.27\textwidth}|p{0.40\textwidth}|p{0.08\textwidth}|p{0.06\textwidth}|}
\caption{Summary of variable definitions, units, and data types for each data set file.
} \\ 
  \hline
{\textbf{Data file}} & {\textbf{Variable identity}} & {\textbf{Variable definition}} & {\textbf{Units}} & {\textbf{Type}} \\ 
  \hline
a. & functional\_group & Functional group for each species. The classification of species into functional groups is based on Moisan et al. (2023). &  & string \\ 
   \hline
a. & scientific\_name & Bird scientific names were obtained from the IOC World Bird List 14.2 (Gill et al., 2024). Mammal scientific names were obtained from the Mammal species of the world: a taxonomic and geographic reference (Wilson, 2005). However, we considered here Mustela erminea richardsonii as Mustela richardsonii due to recent genetic analysis (Colella et al., 2021). &  & string \\ 
   \hline
a. & species & Common species english name. &  & string \\ 
   \hline
a. & annual\_cycle\_strategy & Annual cycle strategy of each species (i.e., resident, partially migratory or migratory). The classification of species annual cycle strategy is based on Gauthier et al., (2011)  and Moisan et al. (2023). &  & string \\ 
   \hline
b. & species & Common species english name. &  & string \\ 
   \hline
b. & year & Year corresponding to the estimate of annual abundance. & years & integer \\ 
   \hline
b. & status & Reproductive status of the individuals (i.e., undetermined or breeding). &  & string \\ 
   \hline
b. & method & Brief overview of the method used to estimate the species abundance. &  & string \\ 
   \hline
b. & abundance & Estimate of the annual number of individuals found within the 389 km2 study area located on the southern part of Bylot Island during the breeding season (May to August). This includes both breeding and non-breeding individuals that stay in the study area for a significant period of time, and excludes non-breeding individuals that stop for only a few days during their migration. The estimates only consider adults, with the exception of lemmings, for which no distinction has been made between juveniles and adults. & individuals & integer \\ 
   \hline
c. & species & Common species english name. &  & string \\ 
   \hline
c. & method & Brief overview of the method used to estimate the species abundance. &  & string \\ 
   \hline
c. & mean\_abundance & Estimate of the mean annual number of individuals found within the 389 km2 study area located on the southern part of Bylot Island during the breeding season (May to August). This includes both breeding and non-breeding individuals that stay in the study area for a significant period of time, and excludes non-breeding individuals that stop for only a few days during their migration. The estimates only consider adults, with the exception of lemmings, for which no distinction has been made between juveniles and adults. & individuals & integer \\ 
   \hline
c. & sd\_abundance & Standard deviation of the annual species abundance. & individuals & numeric \\ 
   \hline
c. & sample\_size\_abundance & Number of years consider in the calculation of mean annual abundance. & years & integer \\ 
   \hline
d. & species & Common species english name. &  & string \\ 
   \hline
d. & source & Source from which estimate were derived (i.e., measurements from the study area or  extracted from Wilman et al. 2014). &  & string \\ 
   \hline
d. & body\_mass\_g & Mean individual body mass. & grams & numeric \\ 
   \hline
e. & species & Common species english name. &  & string \\ 
   \hline
e. & zone & Sampled zone of the study area (see figure 2 and 3). &  & string \\ 
   \hline
e. & mean\_nest\_density\_km2 & Estimate of the mean annual nest density measured within the corresponding zone
of the study area. & nests per square kilometer & numeric \\ 
   \hline
e. & sd\_nest\_density\_km2 & Standard deviation of the annual nest density measured within the corresponding
zone of the study area. & nests per square kilometer & numeric \\ 
   \hline
e. & sample\_size\_nest\_density\_km2 & Number of years consider in the calculation of the nest density. & years & integer \\ 
   \hline
\hline
\label{table:summary_variables}
\end{longtable}
\endgroup
