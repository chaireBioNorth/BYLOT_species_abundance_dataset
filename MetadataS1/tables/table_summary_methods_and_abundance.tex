% latex table generated in R 4.4.1 by xtable 1.8-4 package
% Tue Oct  1 09:17:43 2024
\begingroup\fontsize{8pt}{10pt}\selectfont
\begin{longtable}{|p{0.10\textwidth}|p{0.27\textwidth}|p{0.06\textwidth}|p{0.23\textwidth}|p{0.09\textwidth}|p{0.09\textwidth}|p{0.085\textwidth}|p{0.04\textwidth}|p{0.10\textwidth}|}
\caption{Summary of the lowest, highest, mean and standard deviation of the estimated abundance of each vertebrate species in the vertebrate community of the southern plain of Bylot Island (389 km2). In some cases, two independent approaches have been used to estimate the abundance of the same species as a proxy for uncertainty. We provide a qualitative measure of the method quality based on data available, method used for extrapolation (if necessary), and in some cases, from the fit of statistical models to estimate density.} \\ 
  \hline
{\textbf{Species}} & {\textbf{Method}} & {\textbf{Method quality}} & {\textbf{Justification}} & {\textbf{Lowest abundance}} & {\textbf{Highest abundance}} & {\textbf{Mean abundance}} & {\textbf{sd}} & {\textbf{n}} \\ 
  \hline
Pacific loon & Intensive study area-wide nest monitoring (389 km2) & high & No extrapolation &   0 &   6 & 4 & 3 & 4 (2017-2019, 2022) \\ 
   \hline
Red-throated loon & Intensive study area-wide nest monitoring (389 km2) & high & No extrapolation &  42 &  76 & 64 & 15 & 4 (2017-2019, 2022) \\ 
   \hline
King eider & Intensive, but opportunistic nest monitoring (8 km2) extrapolated by habitat & very low & Intensive, but opportunistic monitoring at relatively small spatial scale, but does not include potential non-breeding individuals &  &  & 25 &  &  \\ 
   \hline
King eider & Derived from the abundance estimate of red-throated loon using incidental observations & low & Derived from high quality estimate of another species &  &  & 106 &  &  \\ 
   \hline
Long-tailed duck & Intensive, but opportunistic nest monitoring (8 km2) extrapolated by habitat & very low & Intensive, but opportunistic monitoring at relatively small spatial scale, but does not include potential non-breeding individuals &  &  & 20 &  &  \\ 
   \hline
Long-tailed duck & Derived from the abundance estimate of red-throated loon using incidental observations & low & Derived from high quality estimate of another species &  &  & 191 &  &  \\ 
   \hline
Cackling goose & Extrapolation from exponential model of growth (R2=0.74, p=0.15, n=5) & moderate & Strong correlation with opportunistic nest monitoring &   2 & 158 & 31 & 41 & 23 (1996-2016, 2020-2021) \\ 
   \hline
Cackling goose & Intensive study area-wide nest monitoring (389 km2) & high & No extrapolation &  80 & 214 & 138 & 50 & 5 (2017-2019, 2022-2023) \\ 
   \hline
Snow goose & Nest monitoring plots extrapolated to mean goose colony area & moderate & Relatively small sample size and uncertainty on goose colony area  & 2505 & 35404 & 18129 & 11037 & 11 (1999-2009) \\ 
   \hline
Snow goose & Intensive study area-wide monitoring based on a combination of methods (transects, point counts and nest monitoring plots) and annual colony outline & high & Multiple independent methods and annual colony outline & 8687 & 49076 & 31852 & 12092 & 12 (2010-2019, 2022-2023) \\ 
   \hline
Tundra swan & Intensive study area-wide nest monitoring (389 km2) & high & No extrapolation &   0 &   2 & 1 & 1 & 4 (2017-2019, 2022) \\ 
   \hline
Rough-legged hawk & Extrapolation from intensive nest monitoring (111 km2, R2=0.99, p$<$0.0001, n=8) & high & Strong correlation with study area-wide nest density &   3 &  59 & 22 & 21 & 6 (2007-2012) \\ 
   \hline
Rough-legged hawk & Intensive study area-wide nest monitoring (389 km2) & high & No extrapolation &   0 &  66 & 30 & 29 & 8 (2013-2019, 2022) \\ 
   \hline
Peregrine falcon & Intensive study area-wide nest monitoring (389 km2) & high & No extrapolation &   8 &  12 & 10 & 1 & 8 (2013-2019, 2022) \\ 
   \hline
Snowy owl & Extrapolation from intensive nest monitoring (111 km2, R2=0.99, p$<$0.0001, n=10) & high & Strong correlation with study area-wide nest density &   0 &  67 & 15 & 24 & 16 (1996-2011) \\ 
   \hline
Snowy owl & Intensive study area-wide nest monitoring (389 km2) & high & No extrapolation &   0 & 144 & 17 & 45 & 10 (2012-2019, 2022-2023) \\ 
   \hline
Rock ptarmigan & Intensive, but opportunistic nest monitoring (8 km2) extrapolated to study area & very low & Intensive, but opportunistic monitoring at relatively small spatial scale and prime nesting habitat not well sampled &  &  & 24 &  &  \\ 
   \hline
Sandhill crane & Extrapolation from intensive nest monitoring (8 km2) and transect observations (R2= 0.44 p= 0.32, n=8) & moderate & Uncertain relation with large scale indices &  &  & 34 &  &  \\ 
   \hline
American golden-plover & Distance sampling throughout lowland (313 km2) & high & Large sample size & 397 & 1725 & 1102 & 432 & 8 (2014-2019, 2022-2023) \\ 
   \hline
Black-bellied plover & Derived from the abundance estimate of American golden-plover using transects observations & low & Derived from high quality estimate of another species &  &  & 29 &  &  \\ 
   \hline
Black-bellied plover & Derived from the abundance estimate of American golden-plover using incidental observations & very low & Derived from high quality estimate of another species, but potentially includes transient migratory individuals &  &  & 87 &  &  \\ 
   \hline
Common Ringed plover & Nest monitoring on the main breeding sites & moderate & Intensive monitoring, but not exhaustive to study area &  44 &  62 & 55 & 9 & 3 (2015-2017) \\ 
   \hline
Ruddy turnstone & Derived from the abundance estimate of Bairds sandpiper using transects observations & low & Derived from moderate quality estimate of another species &  &  & 40 &  &  \\ 
   \hline
Ruddy turnstone & Derived from the abundance estimate of Bairds sandpiper using incidental observations & very low & Derived from moderate quality estimate of another species, but potentially includes transient migratory individuals &  &  & 53 &  &  \\ 
   \hline
Red knot & Derived from the abundance estimate of Bairds sandpiper using transects observations & low & Derived from moderate quality estimate of another species &  &  & 66 &  &  \\ 
   \hline
Red knot & Derived from the abundance estimate of Bairds sandpiper using incidental observations & very low & Derived from moderate quality estimate of another species, but potentially includes transient migratory individuals &  &  & 233 &  &  \\ 
   \hline
Pectoral sandpiper & Derived from the abundance estimate of Bairds sandpiper using transects observations & low & Derived from moderate quality estimate of another species &  &  & 80 &  &  \\ 
   \hline
Pectoral sandpiper & Derived from the abundance estimate of Bairds sandpiper using incidental observations & very low & Derived from moderate quality estimate of another species, but potentially includes transient migratory individuals &  &  & 255 &  &  \\ 
   \hline
Baird's sandpiper & Extrapolation from intensive nest monitoring (2 km2) and transects observations (R2=0.38, p=0.35, n=8) & moderate & Uncertain relation with large scale indices &  &  & 2448 &  &  \\ 
   \hline
White-rumped sandpiper & Derived from the abundance estimate of Bairds sandpiper using transects observations & low & Derived from moderate quality estimate of another species &  &  & 991 &  &  \\ 
   \hline
White-rumped sandpiper & Derived from the abundance estimate of Bairds sandpiper using incidental observations & very low & Derived from moderate quality estimate of another species, but potentially includes transient migratory individuals &  &  & 1134 &  &  \\ 
   \hline
Buff-breasted sandpiper & Derived from the abundance estimate of Bairds sandpiper using transects observations & low & Derived from moderate quality estimate of another species &  &  & 6 &  &  \\ 
   \hline
Buff-breasted sandpiper & Derived from the abundance estimate of Bairds sandpiper using incidental observations & very low & Derived from moderate quality estimate of another species, but potentially includes transient migratory individuals &  &  & 8 &  &  \\ 
   \hline
Red phalarope & Derived from the abundance estimate of Bairds sandpiper using transects observations & low & Derived from moderate quality estimate of another species &  &  & 140 &  &  \\ 
   \hline
Red phalarope & Derived from the abundance estimate of Bairds sandpiper using incidental observations & very low & Derived from moderate quality estimate of another species, but potentially includes transient migratory individuals &  &  & 270 &  &  \\ 
   \hline
Glaucous gull & Extrapolation from intensive nest monitoring (111 km2, R2=0.84, p=0.16, n=4) & high & Strong correlation with study area-wide nest density &  59 &  80 & 73 & 6 & 13 (2004-2016) \\ 
   \hline
Glaucous gull & Intensive study area-wide nest monitoring (389 km2) & high & No extrapolation &  60 &  80 & 71 & 9 & 4 (2017-2019, 2022) \\ 
   \hline
Long-tailed jaeger & Intensive nest monitoring (33 km2) extrapolated by habitat & high & Relatively large spatial coverage of sampling &   0 & 900 & 272 & 285 & 17 (2004-2019, 2022) \\ 
   \hline
Parasitic jaeger & Maximum number of individuals banded in a year & low & Based on a single year and potentially not all individuals were captured &  &  & 17 &  &  \\ 
   \hline
Parasitic jaeger & Maximum number of nest found annually during study area-wide opportunistic nest monitoring & very low & Monitoring does not include potential non-breeding individuals &  &  & 8 &  &  \\ 
   \hline
Common raven & Derived from the abundance estimate of glaucous gull using transects observations & very low & Derived from moderate quality estimate of another species, but potential difference in detectability between species &  &  & 14 &  &  \\ 
   \hline
Common raven & Derived from the abundance estimate of glaucous gull using incidental observations & very low & Derived from moderate quality estimate of another species, but potential difference in detectability between species &  &  & 31 &  &  \\ 
   \hline
Horned lark & Derived from the abundance estimate of Lapland longspur using transects observations & low & Derived from moderate quality estimate of another species &  &  & 362 &  &  \\ 
   \hline
Horned lark & Derived from the abundance estimate of Lapland longspur using incidental observations & low & Derived from moderate quality estimate of another species &  &  & 411 &  &  \\ 
   \hline
American pipit & Derived from the abundance estimate of Lapland longspur using transects observations & very low & Derived from moderate quality estimate of another species and prime nesting habitat not sampled &  &  & 53 &  &  \\ 
   \hline
American pipit & Derived from the abundance estimate of Lapland longspur using incidental observations & low & Derived from moderate quality estimate of another species and prime nesting habitat not well sampled &  &  & 87 &  &  \\ 
   \hline
Lapland longspur & Extrapolation from intensive nest monitoring (2 km2) and transects observations (R2=0.62, p=0.1, n=8) & moderate & Uncertain relation with large scale indices &  &  & 7110 &  &  \\ 
   \hline
Snow bunting & Derived from the abundance estimate of Lapland longspur using transects observations & very low & Derived from moderate quality estimate of another species and prime nesting habitat not sampled &  &  & 18 &  &  \\ 
   \hline
Snow bunting & Derived from the abundance estimate of Lapland longspur using incidental observations & low & Derived from moderate quality estimate of another species and prime nesting habitat not well sampled &  &  & 276 &  &  \\ 
   \hline
Nearctic brown lemming & Rigorous density estimates at small spatial scale (0.22 km2) extrapolated by habitat & moderate & Intensive sampling, but small spatial coverage and extrapolation by habitat &   0 & 447630 & 54043 & 93530 & 27 (1995-2019, 2021-2022) \\ 
   \hline
Nearctic collared lemming & Rigorous density estimates at small spatial scale (0.22 km2) extrapolated by habitat & moderate & Intensive sampling, but small spatial coverage and extrapolation by habitat &   0 & 39302 & 8128 & 10334 & 27 (1995-2019, 2021-2022) \\ 
   \hline
Arctic hare & Derived from the abundance estimate of Arctic fox using incidental observations & very low & Derived from moderate quality estimate of another species, prime nesting habitat not well sampled and difference in detectability between species &  &  & 6 &  &  \\ 
   \hline
Ermine & Indices of relative abundance derived from testimonials converted to abundance using home range size & moderate & Indirect indices and uncertainty on ermine home range size estimates &   8 & 156 & 40 & 37 & 27 (1993-2019) \\ 
   \hline
Arctic fox & Derived from extensive fox home range size studies (n=109) & moderate & Indirect indices, but large sample size &  &  & 53 &  &  \\ 
   \hline
\hline
\label{table:table_summary_methods_and_abundance}
\end{longtable}
\endgroup
