% latex table generated in R 4.4.1 by xtable 1.8-4 package
% Wed Oct 16 14:36:55 2024
\begin{table}[ht]
\centering
\caption{Index of relative abundance (i.e., number of individuals observed per hour) derived from incidental daily observations for selected vertebrate species in lowland (i.e., Qarlikturvik valley, Camp 3, Malaview, Camp 2, Goose point and Pointe Dufour) and upland (i.e., Black plateau, Southern plateau and Camp 3 plateau) zones of the Bylot Island study area. The ratio compares relative abundance indexes between these two types of zones, calculated by dividing the upland by the lowland index of relative abundance.} 
\label{table:table_species_relative_abundance_upland}
\begingroup\fontsize{10pt}{10pt}\selectfont
\begin{tabularx}{0.55\textwidth}{lrll}
  \hline
  & \multicolumn{2}{c}{Individuals/hour} \\
 Species & Upland & Lowland & Ratio \\
 \hline
Rock ptarmigan & 0.03 & 0.03 & 1 \\ 
  Sandhill crane & 0.42 & 0.287 & 1.5 \\ 
  American golden-plover & 0.26 & 0.394 & 0.7 \\ 
  Black-bellied plover & 0.02 & 0.032 & 0.6 \\ 
  Ruddy turnstone & 0.01 & 0.007 & 1.3 \\ 
  Red knot & 0.00 & 0.033 & 0 \\ 
  Pectoral sandpiper & 0.02 & 0.034 & 0.6 \\ 
  Baird's sandpiper & 0.31 & 0.32 & 1 \\ 
  White-rumped sandpiper & 0.04 & 0.137 & 0.3 \\ 
  Buff-breasted sandpiper & 0.00 & 0.001 & 0 \\ 
  Red phalarope & 0.01 & 0.038 & 0.2 \\ 
  Horned lark & 0.24 & 0.154 & 1.6 \\ 
  American pipit & 0.34 & 0.024 & 14.2 \\ 
  Lapland longspur & 1.93 & 2.641 & 0.7 \\ 
  Snow bunting & 0.59 & 0.092 & 6.4 \\ 
  Arctic hare & 0.02 & 0.009 & 2 \\ 
   \hline
\end{tabularx}
\endgroup
\end{table}
