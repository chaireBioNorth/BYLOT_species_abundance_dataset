% latex table generated in R 4.4.1 by xtable 1.8-4 package
% Thu Oct 10 16:08:11 2024
\begingroup\fontsize{8pt}{10pt}\selectfont
\begin{longtable}{|p{0.04\textwidth}|p{0.21\textwidth}|p{0.07\textwidth}|p{0.40\textwidth}|p{0.05\textwidth}|p{0.07\textwidth}|}
\caption{Summary of variable storage type, code definition, range and number of digit.} \\ 
  \hline
{\textbf{Data file}} & {\textbf{Variable identity}} & {\textbf{Storage type}} & {\textbf{Definition variable codes}} & {\textbf{Range}} & {\textbf{Number digits}} \\ 
  \hline
a. & class & string & NA & NA & NA \\ 
   \hline
a. & order & string & NA & NA & NA \\ 
   \hline
a. & family & string & NA & NA & NA \\ 
   \hline
a. & genus & string & NA & NA & NA \\ 
   \hline
a. & species\_scientific & string & NA & NA & NA \\ 
   \hline
a. & species\_en & string & NA & NA & NA \\ 
   \hline
a. & species\_fr & string & NA & NA & NA \\ 
   \hline
a. & functional\_group & string & NA & NA & NA \\ 
   \hline
a. & migratory\_status & string & resident: Individuals performing movements within the study area throughout the annual cycle.; partial migrant: A combination of resident and migratory and/or individuals performing long-distance foraging trips outside the study area during the non-breeding period.; migrant: Individuals performing seasonal and highly synchronous movements between the study area and a distant non-breeding ground. & NA & NA \\ 
   \hline
b. & species\_en & string & NA & NA & NA \\ 
   \hline
b. & year & integer & If abundance has not been calculated for a given series of years, but rather as a general average, then NA has been assigned. & 1993-2023 & 0 \\ 
   \hline
b. & breeding\_status & string & undetermined: Individuals present on the study area during the breeding period (June to August) that might have breed or not.; breeding: Individuals present in the study area during the breeding period (June to August) and having attempted and/or completed breeding. & NA & NA \\ 
   \hline
b. & abundance & integer & NA & 0-447630 & 0 \\ 
   \hline
b. & method\_description & string & NA & NA & NA \\ 
   \hline
b. & method\_quality & string & very low: Sampling might not encompasses prime nesting habitat, excludes transient migratory individuals or includes potential non-breeding individuals. If abundance is derived from the abundance estimate of another species based relative abundance, detection probabilities may differ.; low: Abundance is derived from the estimate of another species based on indices of relative abundance.; moderate: Small to intermediate scale sampling with spatial extrapolation.; high: Large scale intensive sampling, with some spatial extrapolation in a few cases. & NA & NA \\ 
   \hline
c. & species\_en & string & NA & NA & NA \\ 
   \hline
c. & site & string & bylot: Southern plain of Bylot Island, Nunavut, Canada.; undetermined: Data were not retrieved from original publications. & NA & NA \\ 
   \hline
c. & mean\_body\_mass\_g & integer & NA & 21 - 6378 & 0 \\ 
   \hline
c. & sample\_size & integer & NA & 1 - 6405 & 0 \\ 
   \hline
c. & reference & string & NA & NA & NA \\ 
   \hline
d. & species\_en & string & NA & NA & NA \\ 
   \hline
d. & zone & string & qarlikturvik (2x1 km plot): Intensive search plot (2 km2) for Lapland Longspur and Baird's sandpiper nests on the south side of the glacial river in the Qarlikturvik valley.;
qarlikturvik (4x2 km plot): Intensive search plot (2 km2) for Sandhill crane, Long-tailed duck, King eider and Rock ptarmigan nests on the south side of the glacial river in the Qarlikturvik valley.;
qarlikturvik valley:Intensive search area (33 km2) for long-tailed jaeger nests on the south side of the glacial river in the Qarlikturvik valley.; whole study area: Entire study area (389 km2) located on the southern plain of Bylot Island. & NA & NA \\ 
   \hline
d. & mean\_nest\_density\_km2 & numeric & NA & 0.001-13.559 & 3 \\ 
   \hline
d. & sd\_nest\_density\_km2 & numeric & NA & 0.001-5.849 & 3 \\ 
   \hline
d. & number\_years & integer & NA & 3-17 & 0 \\ 
   \hline
\hline
\label{table:table_variables_format}
\end{longtable}
\endgroup
