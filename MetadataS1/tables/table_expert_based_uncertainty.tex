% latex table generated in R 4.4.1 by xtable 1.8-4 package
% Tue Oct  1 09:17:24 2024
\begin{table}[ht]
\centering
\caption{Considering the absence of confidence intervals in our abundance estimates, we present below uncertainty intervals on estimated abundance values derived from field expert impressions. The intervals presented represent the minimum and maximum values between which the experts believe the actual abundance values should lie. When annual abundance has been estimated for several years (time series) we present intervals over the minimum and maximum abundance values encountered during the given time series. For other species, the uncertainty interval is estimated on the mean abundance.} 
\begingroup\fontsize{10pt}{8pt}\selectfont
\begin{tabularx}{\textwidth}{lllll}
  \hline
  & &\multicolumn{3}{c}{Annual abundance (ind.)} \\
 Species & Period & Lowest & Highest & Mean \\
 \hline
Snow goose & 1999-2009 & [2500-10000] & [35000-60000] &  \\ 
  Snow goose & 2010-2019, 2022-2023 & [6000-10000] & [45000-60000] &  \\ 
  Snowy owl & 1996-2011 & 0 & [50-100] &  \\ 
  Snowy owl & 2012-2019, 2022-2023 & 0 & [144-170] &  \\ 
  Glaucous gull & 2004-2016 & [50-80] & [70-100] &  \\ 
  Glaucous gull & 2017-2019, 2022 & [60-80] & [80-100] &  \\ 
  Peregrine falcon & 2013-2019, 2022 & [8-12] & [12-20] &  \\ 
  Rough-legged hawk & 2007-2012 & [0-8] & [50-90] &  \\ 
  Rough-legged hawk & 2013-2019, 2022 & [0-4] & [66-86] &  \\ 
  American golden-plover & 2014-2019, 2022-2023 & [100-500] & [1000-2500] &  \\ 
  Cackling goose & 1996-2016, 2020-2021 & [2-10] & [58-220] &  \\ 
  Cackling goose & 2017-2019, 2022-2023 & [80-110] & [214-244] &  \\ 
  Arctic fox & Mean abundance &  &  & [30-60] \\ 
  Nearctic collared lemming & 1995-2019, 2021-2022 & [100-2000] & [20000-50000] &  \\ 
  Nearctic brown lemming & 1995-2019, 2021-2022 & [100-2000] & [200000-450000] &  \\ 
  Ermine & 1993-2019 & [0-10] & [50-156] &  \\ 
  Long-tailed jaeger & 2004-2019, 2022 & [0-10] & [300-900] &  \\ 
  Red-throated loon & 2017-2019, 2022 & [42-62] & [76-96] &  \\ 
  Pacific loon & 2017-2019, 2022 & [0-6] & [6-10] &  \\ 
  Tundra swan & 2017-2019, 2022 & [0-4] & [2-6] &  \\ 
  Common Ringed plover & 2015-2017 & [44-60] & [60-85] &  \\ 
  Black-bellied plover & Mean abundance &  &  & [6-30] \\ 
  Lapland longspur & Mean abundance &  &  & [7000-10000] \\ 
  Baird's sandpiper & Mean abundance &  &  & [1500-3500] \\ 
  Sandhill crane & Mean abundance &  &  & [15-45] \\ 
  King eider & Mean abundance &  &  & [60-250] \\ 
  Long-tailed duck & Mean abundance &  &  & [80-300] \\ 
  Rock ptarmigan & Mean abundance &  &  & [10-60] \\ 
  Horned lark & Mean abundance &  &  & [200-600] \\ 
  Ruddy turnstone & Mean abundance &  &  & [10-30] \\ 
  Red phalarope & Mean abundance &  &  & [20-80] \\ 
  Red knot & Mean abundance &  &  & [10-30] \\ 
  White-rumped sandpiper & Mean abundance &  &  & [1000-2000] \\ 
  Buff-breasted sandpiper & Mean abundance &  &  & [2-10] \\ 
  Pectoral sandpiper & Mean abundance &  &  & [20-100] \\ 
  Parasitic jaeger & Mean abundance &  &  & [15-50] \\ 
  Common raven & Mean abundance &  &  & [30-75] \\ 
  American pipit & Mean abundance &  &  & [50-300] \\ 
  Snow bunting & Mean abundance &  &  & [100-500] \\ 
  Arctic hare & Mean abundance &  &  & [15-50] \\ 
   \hline
\end{tabularx}
\endgroup
\end{table}
